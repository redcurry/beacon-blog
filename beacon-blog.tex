% We will feature weekly posts from BEACON researchers
% (including faculty, post-docs, grad students, and undergrads)
% describing their work. These posts should be around 800-1000 words,
% and written in a non-technical way that would be easily understood
% by an undergraduate non-major. Ideally, the posts should also include
% a personal element, describing your experience or why you are interested
% in your research topic. The goal is not only to communicate BEACON
% research to the public, but also to connect faces and personalities to
% that research. Please include one or two pictures, of something related
% to your research or of you, with your submission, in file formats
% suitable for uploading to the internet (jpg, tiff, bmp).

\documentclass[12pt]{article}

\title{Speciation in digital organisms}
\author{Carlos Anderson}

\begin{document}

\maketitle

% Double-spaced
\baselineskip 24pt



That the structure and laws of our universe
enabled the origin of life is an incredible coincidence.
%
Without the gravity that aggregates matter into galaxies, stars, and planets,
or without the light that radiates from the sun and heats the Earth,
or without the chemical bonds between atoms that store and release energy,
life, as we know it, would have probably never arisen.
%
But would life, as we \emph{don't} know it, have?



I thought about this question in the summer of 2003,
after reading a curious novel by Michael Crichton called \emph{Prey}.
%
In it, scientists create self-reproducing artificial nanobots,
which evolve out of control and prey on their creators
(most of them die, of course).
%
But the concept of an artificial life fascinated me,
and I soon began reading on evolutionary computation,
starting with a wonderful book by Melanie Mitchell.
%
She describes a class of algorithms
that use the principles of evolution by natural selection%
---inheritance, variation, and differential reproduction---%
to search for solutions to various kinds of problems,
such as the design of efficient engines
or the sequence reconstruction of the human genome.
%
These `genetic algorithms' demonstrate that evolution
is not unique to biological life
but to any system having certain basic properties.



With this new interest in evolution I began taking classes in biology,
and although I had recently obtained a B.S. degree in Computer Science,
I decided to pursue an M.S. degree in Biology.
%
%Back in 2004, I had asked one of my biology professors how
%I can apply my computer skills to solving biological problems,
%and she happened to have been looking
%for someone like me to help her solve a problem.
%
%She studied polar behavior and needed a way to
%identify individuals, but because tagging them was too invasive,
%she wanted to identify them from photographs.
%
%She had the idea of using whisker spot patterns,
%as had been done in lions,
%but because comparing whisker spot patterns is tedious,
%she wanted to automate the process.
%
%Such computer-aided identification system became my thesis.
%
%It wasn't exactly evolutionary,
%nor revolutionary for that matter,
%but the Masters allowed me to take
%graduate classes in evolution
%and develop a specific interest within evolution.
This first exposure to graduate school
helped me transition from by background
in computer science to biology,
and it allowed me to develop 
a specific interest within evolution
that I could study further as a dissertation.



This interest was speciation, the process by which new species arise.
%
There are many definitions of `species,' some based on morphology,
others on ecology, and still others on genetic differences.
%
But the one most widely accepted is the biological species concept,
in which populations are considered to be different species
if they are reproductively isolated;
that is, if they cannot produce fertile or viable offspring,
either because they are unable to mate
or because their hybrid offspring are sterile or inviable.
%
Reproductive isolation is thought to evolve most readily
when populations become geographically isolated and
each subpopulation adapts independently to its local environment.
%
Two big questions in speciation are
(1) can speciation proceed even if environments are similar to each other,
and (2) can speciation proceed when migration between populations occurs?
%
I wanted to study speciation in an artificial life system,
where my findings could be generalized to life as we don't know it.



Searching for Ph.D. programs in 2006, I found Michigan State University,
where a team of scientists and students have been developing Avida.
%
Avida is an artificial life software
designed to study questions in evolution and ecology.
%
In Avida, digital organisms consist of a sequence of instructions (or `genome')
that encodes their ability to replicate and perform computational functions.
%
The precise sequence of instructions that
allow organisms to perform functions
evolve through natural selection and genetic drift,
two evolutionary processes that occur in biological organisms.
%
With Avida, one can observe millions of generations
of evolution in a short period of time,
perform many replications, easily manipulate genomes,
and accurately record measurements like fitness and events like mutations.



One of my studies addresses whether speciation
can occur when environments are similar between populations.
%
One hypothesis is that speciation can happen by compensatory adaptation,
in which a deleterious mutation rises in frequency in a population
and is subsequently compensated by secondary mutations.
%
Imagine that two populations become divided
and each undergoes the process of compensatory adaptation.
%
If the populations were now to come into contact,
their hybrids would inherit a combination of
deleterious and compensatory mutations,
which, because they evolved independently,
may not be compatible with each other
and possibly cause inviability or sterility.
%
I found this hybrid incompatibility in Avida,
and it wasn't simply because hybrids inherited
deleterious mutations, but also because
compensatory mutations between populations were incompatible.
%
These findings show that compensatory adaptation is
one way in which speciation can occur when
the external environment does not change.



Another of my studies tests the effect of
migration between diverging populations on the probability of speciation.
%
I evolved populations that had to
adapt to a new environment while
migration between them occurred.
%
Although the environments were new,
I had treatments in which the two environments
were different from each other
and in which they were the same.
%
I found that when the environments are different,
migration does not prevent speciation from starting%
---even at 10\% migration.
%
However, when the environments are the same,
even 1\% migration prevents speciation.
%
It appears that when the environments are the same,
the population that adapts to it first
and lets an individual migrate to the other population,
effectively gives away its solution.
%
This causes both populations to adapt similarly,
preventing reproductive isolation between them.



Interestingly, these findings agree with theoretical expectations.
%
Such unsurprising results are actually good
because they show that Avida behaves like biology,
and therefore demonstrates that
evolution does not require the intricacies of biological life.
%
Although Avida was not designed to study the \emph{origin} of life,
it does make one ponder whether artificial life could evolve \emph{de novo},
and thus show that life can emerge
not from specific universal structures or laws
but from general ones.



\end{document}
